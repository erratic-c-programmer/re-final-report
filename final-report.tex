\documentclass{article}
\usepackage{amsmath}
\usepackage{graphicx}
\usepackage[colorlinks=true,citecolor=.,linkcolor=.]{hyperref}
\usepackage{microtype}
\usepackage[table]{xcolor}
\usepackage{apacite}
\usepackage{multicol}
\usepackage{pdfpages}
\usepackage[toc,page]{appendix}
\usepackage[capitalise]{cleveref}


\begin{document}
\begin{titlepage}
  \begin{center}
    \Large{Raffles Institution \\ Year 2 Research Education \\ Final report} \\
    \vspace{1cm}
    \huge{How to harness green roof technology in schools to encourage
      Singaporean students, through CCAs, to play an active role in reducing
      carbon footprints and controlling the negative effects of greenhouse
    gases} \\
    \vspace{1cm}
    \large{
      \textbf{Authors}: \\
      Nathaniel Chong(3) \\
      Lee Jun Wei(13)\footnote{Group leader} \\
      Teng Zi Huan(28) \\
      Isaac Yeo(32) \\
      \vspace{1cm}
      \begin{tabular}{r@{:}l}
        \textbf{Class} & \hspace{1cm} 2F \\
        \textbf{Teacher} & \hspace{1cm} Dr. Tan Guoxian \\
      \end{tabular}
    }
  \end{center}
\end{titlepage}
\newpage

\begin{abstract}
  \noindent{
    In Singapore, many students do not play fully understand climate
    change and the need to protect the environment. Thus, this study
    seeks to investigate the feasibility of educating Singapore youth
    about the environment and encouraging them to play an active role in
    environmental protection. Surveys were conducted, and most respondents
    were secondary school students. Two interviews were also conducted
    with two interviewees from secondary school to gain insight into the
    matter. After further analysis, it was observed that most did not
    really know about green roofs, but a positive perception of green
    roofs. Thus, active participation should be possible with more
    education about green roofs and the severity of climate change.
  }
\end{abstract}

\newpage

\tableofcontents
\newpage

\begin{multicols}{2}

  \section{Introduction}
  \subsection{On green roofs}
  \paragraph{Definition} Green roofs involve growing plants on roofs,
  which can be sorted into muscinal roofs, herbaceous roofs and arbustive
  roofs \cite{ecoeng}.


  \subsection{Benefits}
  \paragraph{} They can reduce energy demand on space conditioning,
  help in purifying air, and if widely adopted, could reduce the urban
  heat island effect, among other benefits \cite{energeff} which will
  be discussed later in this report in \autoref{sec:benef}.


  \subsection{Adoption}
  \paragraph{} If adopted in schools by having students to take care
  of and keep up the green roofs, the students would then build this
  habit of caring for and maintaining a green roof, something they would
  hopefully continue to do upon reaching adulthood in the future when
  they would be able to realise human impact on the environment. This is
  even more important should they become national leaders, who have the
  power to influence the lives of other people. Having this influence
  from young, they would then turn to such technologies which can affect
  climate change for the better.

  \subsection{Target Audience}
  \paragraph{} We chose secondary school CCAs as our target audience as
  they are old enough and mature enough to understand the implications
  of global warming and climate change, and global warming. They will
  also be the leaders of tomorrow, so it is even more important for them
  to understand this.


  \subsection{Purpose and significance of research question}
  \paragraph{Purpose} To find ways to, through green rooftop technology,
  increase Singapore students' awareness of climate change and how they
  can play a part in reducing it. See below for more information as to
  how this topic is relevant to today's dynamic and modern society.

  \paragraph{Significance} Climate change is of significant
  importance. According to \citeA{nasa}, at the rate of climate
  change we are at, the sea levels worldwide would increase by 1--4
  feet. This leads to the question of whether Singapore would truly
  be safe in the future. This thus draws the necessary attention and
  action of the Singapore Government and the citizens. Actions required
  includes educating the youth of the society of the consequences of
  climate change and the possible course of action.  However, students
  in Singapore have ``major gaps in their understanding [of climate
  change]"\cite{student_carbon_footprint}. Therefore, it is necessary
  for us to research methods that can be used to raise awareness of
  climate change in Singaporean students. At the same time, we believe
  that green roofs can be an effective measure in fulfilling its purpose
  in combating climate change and in motivating the Singaporean youth
  to play an active role in it. Green roofs are a potential way to not
  only encourage the next generation of Singaporeans to take climate
  action,  when the country is in their hands. Proper education of the
  youth would, hopefully, eventually lead to a rise in green technology
  and build a greener, healthier world for everyone to live in, one that
  is possibly freed of the grasps and struggles of climate change. Even
  in Singapore, green roofs have been utilised on buildings such as
  the Nanyang Technological University’s School of Art, Design and
  Media. According to \citeA{greenbuild_advant1}, green roofs not only
  play a part in helping to slow climate change, they also help create
  a better environment for residents.


  \section{A closer look at the benefits of green roofs} \label{sec:benef}
  \subsection{Economic benefits}
  \paragraph{} Research on green roof technologies has so far proven them
  beneficial, with \citeA{energeff} mentioning that they can reduce energy
  demand on space conditioning and decrease temperature fluctuations. This
  is agreed on by \citeA{CFGRSG} who states that increased roof insulation
  could reduce space conditioning required in the building. Other
  sources also mention the decreased carbon emissions due to lower energy
  consumption from improved thermal performance \cite{CommAwareGBSyd}
  which reduces cost of energy as there will be up to a 75 percent
  decrease in energy usage for cooling the building, with daily averages
  dropping from 7.5kWh to 1.5kWh \cite{energeff}. \citeA{energeff} also
  mentions that daily temperature fluctuations on roofing membranes are
  significantly reduced, which can increase the lifespan of the roof.


  \subsection{Environmental benefits}
  \paragraph{}
  \citeA{energeff} states that if green roof technologies are
  widely adopted, they could reduce the urban heat island effect
  (a situation where an urban area has higher temperatures
  than surrounding rural areas) by having the plants on the
  green roofs absorb some of the heat. \citeA{HKGreenRoofGL}
  also suggests that it “mitigates the urban heat island
  effect”. It is also said that green roofs can increase the
  aesthetics of urban landscape, reduce glare for surrounding
  buildings, showing its importance and relevance in today's
  highly urbanised society. Additionally, \citeA{HKGreenRoofGL}
  found that green roofs can mitigate air quality issues, which
  are important for the wellbeing of all. Therefore, we find
  that there is a need to educate students, especially those in
  secondary institutions (as they are mature enough to understand
  the gravity of global warming and climate change and the need
  to take immediate action, and are more likely to have time to
  undertake this project than those in tertiary institutions)
  about green roofs.  Vegetation on green roofs help purify the
  air and convert carbon dioxide into oxygen, which reduces
  the amount of greenhous gases in the air. \cite{energeff}
  mentions this, and \citeA{CommAwareGBSyd} goes a step further,
  even suggesting that green roofs help achieve zero carbon
  footprints. The plants also take in rainwater, reducing the
  water in the sewage system which needs to be purified and
  discharged to the sea, helping to stabilize the groundwater
  level and reducing the possibility of the sewer clogging and
  malfunctioning.



  \section{Methodology}
  \subsection{Data collection}
  \paragraph{Purpose} We believe that green roofs are severely underused
  in Singapore despite the advantages, and think that schools are a
  great place to have them implemented. The surveys and interview we
  conducted were in order to find out Singaporean students’ awareness
  and perception of green roofs, as well as the viability of, and their
  willingness to assist in green roof projects in schools.

  \paragraph{Interviews} Two interviews were conducted. To have more
  accurate and fair data, we chose two interviewees from two different
  schools and different genders. We carried out the interview on 28 June
  3pm. As both interviewees were unable to use Microsoft Teams due to a
  lack of access, we used other means to conduct the interview, such as
  Zoom or Discord video calls. The first interviewee was a girl studying
  in Nanyang Girl High, and the second was a boy studying in West Spring
  secondary. This is to ensure diversity and fairness as both parties
  may have certain bias held against green roofs.

  \paragraph{Surveys} The survey respondents came from two different
  age groups: primary and secondary school students. In total, we
  received survey responses from 21 students, one of which was not
  a serious response (evident from the options selected, which were
  all falling in the ``Strongly disagree", ``False", or similar
  categories, even disagreeing to the PDPA clause). Hence, we chose to
  omit the data gathered from that respondent and focus on the other
  20 respondents. However, 19 of the remaining 20 respondents indicated
  that they were of 13--17 years of age, and hence the demographics of our
  survey were quite limited and the data collected may (unfortunately)
  not be representative of the entire student population in Singapore.


  \section{Results}
  \paragraph{Summary} In general, we found that the majority of
  Singaporean students are supportive of the concept of green roofs.

  \subsection{Interpretation}
  \paragraph{} These results showed that in general, teenagers in
  Singapore were supportive of the idea of green roofing. When asked about
  the practical usage of green roofs, 95\% of respondents agreed on all
  aspects that they believed green roof systems being implemented would
  be a good idea. The one respondent that did not agree on every aspect
  noted that building green roofs was a waste of effort, but agreed on
  all other parts. 

  \subsection{Analysis}
  \paragraph{} We noticed that despite the varying levels of understanding
  of green roof systems, there was no clear and specific pattern
  where acceptance and questions regarding the use of green roofs was
  concerned. The questions to gauge their understanding were based on
  two research papers by \citeA{HKGreenRoofGL} and \citeA{energeff},
  but respondents often showed varying degrees of agreement across
  those aspects listed. This suggests that despite their indication of
  their understanding of green roofs, their actual understanding may
  be different, or due to their individual perceptions. Hence, it was
  difficult to come up with sub group analysis.  Interestingly however, we
  also noticed that when comparing the responses of people who indicated
  that they were only 'vaguely familiar' with green roofs were generally
  more optimistic about than ones who indicated that they were 'familiar'
  with them. Unsurprisingly, though, the single respondent who indicated
  that he was "knowledgable" about green roofs was the most optimistic,
  fully agreeing to most of the statements.



  \section{Social factors regarding green roofs}
  \subsection{General perception of green roofs}
  \paragraph{} \citeA{CommAwareGBSyd} found that 55 percent of respondents
  to a survey conducted “strongly agreed” that greenery was important
  and its benefits outweighed its additional costs, showing public support
  of this idea suggesting that our idea may be quite welcome. This is
  supported by \citeA{CFGRSG}, who said that with therapeutic value of
  greenery in reducing stress already established, green roofs are used
  commonly in Singapore to soften the harsh urban environment and to
  improve the quality of life. Modern green roofs are now mainly based on
  cost, energy, water savings and carbon reduction \citeA{CFGRSG}.  71.8
  percent of respondents in a survey conducted by \citeA{CommAwareGBSyd}
  stated that greenery would make a place more attractive to live in
  meaning that greenery could have a positive impact on people’s
  lives, especially for students who spend time in school. However,
  \citeA{GRBuildEnSave} also found that 33.8 percent of respondents to
  their survey had less than a general understanding of the concept of
  green roofs--- with the oldest(76+) and youngest(12--17) age groups
  showing the least understanding, further stressing the need to educate
  students about green roofs, although this may be inaccurate due to
  the small sample size. However, \citeA{student_carbon_footprint}, also
  found that students in Singapore do not fully understand climate change.



  \section{Possible implementation methods}
  \paragraph{} We propose that students be grouped according to three
  different CCA groups: sports, performing arts and uniformed groups,
  with each group doing its own part. The students in sports CCAs
  and uniformed groups will carry out the physically more demanding
  tasks (e.g.doing the actual planting itself and taking part in
  maintenance with the assistance of professionals, etc.) since they
  are more used to carrying out such physical tasks than the students in
  performing arts CCAs. The students in performing arts CCAs can manage
  the logistics(planning projects, budgets, etc.) seeing that they are
  likely to be less physically inclined while everyone still keeping to
  the recommended guidelines by \citeA{HKGreenRoofGL}.



  \section{Conclusion}
  \paragraph{}
  Current research has shown that green roofs can be beneficial by
  maintaining a stable temperature, reducing space conditioning,
  amongst other benefits. However, there is still a knowledge
  gap where Singaporean students’ perceptions of green roofs
  and its viability in schools are not addressed which is why
  our research is relevant and necessary.

\end{multicols}

\bibliographystyle{apacite}
\bibliography{final-report}

\begin{appendices}
  \section{Research project timeline}
  \begin{table}[h!]
    \begin{center}
      \caption{Our Timeline}
      \label{tab:timeline}
      \begin{tabular}{|c|c|}
        \rowcolor{cyan}
        \textbf{Time} & \textbf{Goal} \\ \hline
        T1W5--T1W9 & Group Project Proposal \\ \hline
        \rowcolor{gray}
        T1W9--T2W3 & Literature review \\ \hline
        T2W3--T2W7 & Finish finalised survey \\ \hline
        \rowcolor{gray}
        T2W7--T3W1 & Administer survey \\ \hline
        T3W1--T3W4 & Data analysis \\ \hline
        \rowcolor{gray}
        T3W3--T3W9 & Prepare for oral assessment \\ \hline
        T3W4--T3W10 & Finalise written report \\ \hline
      \end{tabular}
    \end{center}
  \end{table}
\end{appendices}


\end{document}
